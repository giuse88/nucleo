\chapter{Introduzione}
\label{chap:Intro}
\section{Obiettivo}
L'elaborato si propone di descrivere in maniera generale la realizzazione del supporto per il file system in un nucleo multiprogrammato.
\section{Struttura dell'elaborato}
La relazione si compone di 6 capitoli e un'appendice. 
Nel capitolo \ref{cap:FileSystem} si effettua una panoramica generale sul concetto di file system e sulle sue caratteristiche generali.\\
Nel capitolo \ref{cap:Fat}, si descrive con particolare cura il funzionamento del FAT32, mettendo in evidenza le strutture dati di cui è composto e le procedure necessarie per un suo corretto funzionamento. 
Dopo l'introduzione del concetto di file sytem e FAT32, nel capitolo  \ref{cap:Requisiti}, si presentano gli obiettivi del software evidenziando il punto vista dell'utente finale. In questo capitolo analizziamo e sintetizziamo i requisiti utente e software. 
Gli ultimi due capitoli trattano la realizzazione del file system da un punto di vista progettuale. Nel capitolo \ref{cap:Modifiche} si mettono in evidenza tutte le modifiche che sono state introdotte nel nucleo. 
L'ultimo capitolo,infine, individuato dal numero \ref{cap:Struttutra_Progetto}, presenta l'architettura software generale del sistema, mettendo in evidenza i moduli di cui è composto. 

\section{Come Leggere l'elaborato}
Prima di affrontare la lettura di questo elaborato bisogna evidenziare alcuni concetti che verranno usati spesso nell'elaborato. 
\begin{itemize}
  \item Per nucleo si intende il nucleo didattico del Prof. Frosini e Prof. Lettieri, reperibile presso http://calcolatori.iet.unipi.it/. 
  \item Per utente si intende un utilizzatore del nucleo sopracitato. Si assume che l'utente abbia competenze necessarie all'avvio del nucleo e alla scrittura di programmi per questo. 
  \item Il termine sistema si riferisce al sistema file system FAT 32 oggetto della tesi. 
  \item Quando si fa riferimento al file system si intende sempre il FAT 32.  
  \item I termini Partizione e Volume verranno usati come sinonimi. 
\end{itemize}

   